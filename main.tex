\documentclass[11pt, a4paper]{report} % Use 'report' for chapters and a Table of Contents

% ====================================================================
% 1. Preamble & Packages
% ====================================================================

% German language support
\usepackage[ngerman]{babel}  % German language settings
\usepackage[utf8]{inputenc}  % UTF-8 encoding for umlauts
\usepackage[T1]{fontenc}     % Better font encoding

% Essential Math Packages for Linear Algebra
\usepackage{amsmath}   % Provides core math environments (e.g., align)
\usepackage{amssymb}   % Provides math symbols (e.g., \mathbb{R}, \mathbb{C})
\usepackage{mathtools} % Enhances amsmath
\usepackage{multicol}
\usepackage{enumitem}

% Document Layout and Appearance
\usepackage[margin=1in]{geometry} % Sets comfortable margins
\usepackage{fancyhdr}  % For customizing headers/footers (optional, but nice)
\usepackage{graphicx}  % For including images/diagrams (if you add them later)

% Hyperlinks and clickable Table of Contents
\usepackage{hyperref}
\hypersetup{
    colorlinks=true,       % Links werden farbig dargestellt (nicht als Box)
    linkcolor=blue,        % Farbe für interne Links (ToC, Referenzen)
    urlcolor=blue,         % Farbe für URLs
    citecolor=blue,        % Farbe für Zitate
    pdftitle={Lineare Algebra - Zusammenfassung},
    pdfauthor={Paul Geser},
    pdfsubject={Lineare Algebra},
    bookmarks=true,        % PDF Bookmarks aktivieren
    bookmarksopen=true,    % Bookmarks im PDF Reader geöffnet anzeigen
    pdfpagemode=UseOutlines % PDF öffnet mit Bookmarks-Panel
}

% Set up custom header/footer (optional)
\pagestyle{fancy}
\fancyhead[L]{Lineare Algebra - Zusammenfassung}
\fancyhead[R]{\leftmark} % Prints the current chapter name
\fancyfoot[C]{\thepage}
\renewcommand{\headrulewidth}{0.4pt}
\renewcommand{\footrulewidth}{0pt}

% Custom Command for Vectors (makes code cleaner)
\newcommand{\vect}[1]{\mathbf{#1}}

% Remove "Kapitel" prefix from chapter titles and reduce spacing
\usepackage{titlesec}
\titleformat{\chapter}[hang]
{\normalfont\huge\bfseries}{\thechapter}{1em}{}
\titlespacing*{\chapter}{0pt}{0pt}{10pt}

% ====================================================================
\begin{document}
% ====================================================================

% ----------------------------------------------------
% 2. Deckblatt (Title Page)
% ----------------------------------------------------
\begin{titlepage}
    \centering
    \vspace*{1in}
    {\Huge \textbf{Zusammenfassung} \\ \vspace{0.5cm}}
    {\Huge \textbf{Modul: Lineare Algebra} \par}
    \vspace{2cm}

    {\Large Eine Formel- und Konzeptsammlung \par}
    \vspace{4cm}

    {\Large Herbstsemester 2025 \par}
    \vfill

    {\large Erstellt mit \LaTeX{} und \texttt{amsmath} \par}
\end{titlepage}

\clearpage % Start a new page

% ----------------------------------------------------
% 3. Table of Contents
% ----------------------------------------------------
\setcounter{tocdepth}{0}
\tableofcontents

\clearpage

% ----------------------------------------------------
% 4. Chapters (The Core Content)
% ----------------------------------------------------

% Include all chapters from the chapters directory
\chapter{Kompakt Formelsammlung}

\begin{multicols*}{2}

% --------------------------------------------------------
\section{Allgemeine Grundlagen}
% --------------------------------------------------------
\begin{itemize}[leftmargin=*, itemsep=2pt, parsep=0pt, topsep=2pt]
    \item \textbf{Mitternachtsformel} ($ax^2 + bx + c = 0$): \\
    $ \displaystyle
    x_{1,2} = \frac{-b \pm \sqrt{b^2 - 4ac}}{2a}
    $
    \item \textbf{pq-Formel} (normiert: $x^2 + px + q = 0$): \\
    $ \displaystyle
    x_{1,2} = -\frac{p}{2} \pm \sqrt{\left(\frac{p}{2}\right)^2 - q}
    $
    \item \textbf{Skalarprodukt} ($x^T y$): \\
    $ \displaystyle
    \begin{pmatrix} x_1 & x_2 \end{pmatrix} \cdot 
    \begin{pmatrix} y_1 \\ y_2 \end{pmatrix} = 
    x_1 y_1 + x_2 y_2
    $
    \item \textbf{Norm (Länge):} $\|x\| = \sqrt{x^T x}$
    \item \textbf{Winkel $\alpha$:} $\cos(\alpha) = \frac{x^T y}{\|x\| \cdot \|y\|}$
    \item \textbf{Orthogonalität:} $x \perp y \iff x^T y = 0$
    \item \textbf{Multiplikation} (Zeile $\cdot$ Spalte): \\
    $ \displaystyle
    \begin{pmatrix} a & b \\ c & d \end{pmatrix} 
    \begin{pmatrix} e & f \\ g & h \end{pmatrix} = 
    \begin{pmatrix} ae+bg & af+bh \\ ce+dg & cf+dh \end{pmatrix}
    $
    \item \textbf{Transponieren} ($(AB)^T = B^T A^T$): \\
    $ \displaystyle
    \begin{pmatrix} a & b \\ c & d \end{pmatrix}^T = 
    \begin{pmatrix} a & c \\ b & d \end{pmatrix}
    $
    \item \textbf{Determinante} ($\det(A) \neq 0 \iff$ invertierbar): \\
    $ \displaystyle
    \det \begin{pmatrix} a & b \\ c & d \end{pmatrix} = ad - bc
    $ 
    \item \textbf{Inverse} ($(AB)^{-1} = B^{-1} A^{-1}$): \\
    $ \displaystyle
    \begin{pmatrix} a & b \\ c & d \end{pmatrix}^{-1} = 
    \frac{1}{ad-bc}
    \begin{pmatrix} d & -b \\ -c & a \end{pmatrix}
    $
    \item \textbf{Symmetrisch:} $A = A^T$
    \item \textbf{Orthogonal ($Q$):} $Q^T = Q^{-1}$ (Spalten sind orthonormal).
    \item \textbf{Einheit V:} $\mathbf{V} \cdot \mathbf{V}^{-1} = \mathbf{I}, \quad \mathbf{V}^{-1} \cdot \mathbf{V} = \mathbf{I}$

\end{itemize}

\end{multicols*}
\chapter{Vektoren, Matrizen und Gleichungssysteme}

Kommt bald...

\chapter{Gauss-Algorithmus}

Kommt bald...

\chapter{LU-Zerlegung}

Kommt bald...

\chapter{Gauss-Jordan-Elimination}

Kommt bald...

\chapter{Vektorprodukt und Lineare Abbildungen}


\section{Vektorprodukt (Kreuzprodukt)}

\textbf{Definition} \\
Das Vektorprodukt $u \times v$ zweier Vektoren $u, v \in \mathbb{R}^3$ ist ein Vektor $x$, der senkrecht (orthogonal) auf beiden Vektoren steht.

\textbf{Berechnung} \\
Für $u = \begin{pmatrix} u_1 \\ u_2 \\ u_3 \end{pmatrix}$ und $v = \begin{pmatrix} v_1 \\ v_2 \\ v_3 \end{pmatrix}$ gilt:
$$
u \times v = \begin{pmatrix} u_2 v_3 - u_3 v_2 \\ u_3 v_1 - u_1 v_3 \\ u_1 v_2 - u_2 v_1 \end{pmatrix}
$$
Alternativ kann die Regel von Sarrus mit den Einheitsvektoren $e_1, e_2, e_3$ verwendet werden:
$$
u \times v = \det \begin{pmatrix} e_1 & e_2 & e_3 \\ u_1 & u_2 & u_3 \\ v_1 & v_2 & v_3 \end{pmatrix}
$$

\textbf{Eigenschaften}
\begin{itemize}
	\item \textbf{Länge:} $|u \times v| = |u| \cdot |v| \cdot \sin(\varphi)$, wobei $\varphi$ der eingeschlossene Winkel ist. Geometrisch entspricht dies der Fläche des von $u$ und $v$ aufgespannten Parallelogramms.
	\item \textbf{Orthogonalität:} Das Produkt steht senkrecht auf $u$ und $v$ ($u \cdot (u \times v) = 0$).
	\item \textbf{Antikommutativität:} $u \times v = -(v \times u)$.
	\item \textbf{Parallelität:} Sind $u$ und $v$ parallel, ist $u \times v = 0$.
\end{itemize}

\section{Lineare Abbildungen}

\textbf{Definition} \\
Eine Abbildung $F: V \to W$ heisst \textbf{linear}, wenn für alle Vektoren $u, v$ und Skalare $s$ gilt:
\begin{itemize}
	\item $F(u + v) = F(u) + F(v)$ (Additivität)
	\item $F(s \cdot u) = s \cdot F(u)$ (Homogenität)
\end{itemize}

\textbf{Matrix-Darstellung} \\
Jede lineare Abbildung im $\mathbb{R}^n$ lässt sich durch eine Matrix $A$ darstellen.
$$ F(x) = A \cdot x $$
Dabei sind die Spalten der Matrix $A$ die Bilder der Einheitsvektoren $e_1, \dots, e_n$ unter der Abbildung.
$$ A = \begin{pmatrix} | & | & & | \\ F(e_1) & F(e_2) & \dots & F(e_n) \\ | & | & & | \end{pmatrix} $$

\section{Beispiele (2D-Transformationen)}

Hier sind die Standardmatrizen für Transformationen im $\mathbb{R}^2$:

\textbf{1. Skalierung} \\
Streckung um Faktor $\lambda$ (zentrisch am Ursprung):
$$ A = \begin{pmatrix} \lambda & 0 \\ 0 & \lambda \end{pmatrix} $$

\textbf{2. Drehung} \\
Drehung um den Ursprung um den Winkel $\varphi$ (gegen den Uhrzeigersinn):
$$ R_{\varphi} = \begin{pmatrix} \cos \varphi & -\sin \varphi \\ \sin \varphi & \cos \varphi \end{pmatrix} $$

\textbf{3. Spiegelung}
\begin{itemize}
	\item An der x-Achse: $ \begin{pmatrix} 1 & 0 \\ 0 & -1 \end{pmatrix} $
	\item An der y-Achse: $ \begin{pmatrix} -1 & 0 \\ 0 & 1 \end{pmatrix} $
	\item An einer Ursprungsgeraden mit Winkel $\varphi$:
	$$ S_{\varphi} = \begin{pmatrix} \cos(2\varphi) & \sin(2\varphi) \\ \sin(2\varphi) & -\cos(2\varphi) \end{pmatrix} $$
\end{itemize}

\textbf{4. Scherung} \\
Scherung entlang der x-Achse mit Faktor $s$:
$$ A = \begin{pmatrix} 1 & s \\ 0 & 1 \end{pmatrix} $$

\section{Zusammensetzung von Abbildungen}

Werden mehrere lineare Abbildungen nacheinander ausgeführt, entspricht dies der Multiplikation ihrer Matrizen.

\textbf{Reihenfolge} \\
Sei $A$ die erste Abbildung ($x \to Ax$) und $B$ die zweite Abbildung ($y \to By$). Die Gesamtabbildung $C$ ist:
$$ C = B \cdot A $$
\textbf{Wichtig:} Die zuerst ausgeführte Abbildung steht in der Multiplikation rechts (nahe am Vektor $x$).
$$ z = B(Ax) = (B \cdot A)x $$
Die Matrixmultiplikation ist im Allgemeinen nicht kommutativ ($B \cdot A \neq A \cdot B$), daher ist die Reihenfolge entscheidend.

\chapter{Orthogonale Abbildungen und Homogene Koordinaten}

Kommt bald...

\chapter{Einfache lineare Regression}

\section{Die beste lineare Approximation}

\textbf{Problemstellung} \\
Gegeben sind Datenpunkte $(t_i, b_i)$. Gesucht ist eine Gerade $b = C + Dt$, die diese Punkte bestmöglich annähert.
Das resultierende lineare Gleichungssystem $Ax = b$ ist in der Regel überbestimmt ($m > n$, mehr Gleichungen als Unbekannte) und besitzt keine exakte Lösung, da der Vektor $b$ nicht exakt im Spaltenraum $C(A)$ liegt.

\textbf{Lösungsansatz (Methode der kleinsten Quadrate)} \\
Da der Fehler $e = b - Ax$ nicht Null sein kann, suchen wir ein $\hat{x}$, das die Länge des Fehlervektors $\|e\|$ (bzw. $\|e\|^2$) minimiert.
Geometrisch betrachtet ist der Punkt $p = A\hat{x}$, der am nächsten zu $b$ liegt, die \textbf{orthogonale Projektion} von $b$ auf den Spaltenraum $C(A)$.

\section{Orthogonale Projektion}

\subsection{Projektion auf einen Vektor}
Die Projektion eines Vektors $b$ auf einen Vektor $a$ ist gegeben durch:
$$ p = \frac{a^T b}{a^T a} \cdot a $$
Die zugehörige Projektionsmatrix $P$ lautet:
$$ P = \frac{1}{a^T a} a a^T $$

\subsection{Projektion auf einen Unterraum (Spaltenraum)}
Die Projektion von $b$ auf den Spaltenraum einer Matrix $A$ liefert den Vektor $p = A\hat{x}$.
Der Fehlervektor $e = b - p$ muss orthogonal zu allen Spalten von $A$ stehen. Daraus folgt die Bedingung $A^T e = 0$, was zu den \textbf{Normalengleichungen} führt:
$$ A^T (b - A\hat{x}) = 0 \quad \implies \quad A^T A \hat{x} = A^T b $$

\textbf{Lösung für $\hat{x}$:} \\
Wenn die Spalten von $A$ linear unabhängig sind, ist $A^T A$ invertierbar und es gilt:
$$ \hat{x} = (A^T A)^{-1} A^T b $$
Die Projektionsmatrix $P$, die $b$ auf den Spaltenraum abbildet ($p = Pb$), ist:
$$ P = A (A^T A)^{-1} A^T $$

\section{Allgemeines Vorgehen bei der Regression}

Um die Parameter $\hat{x} = \begin{pmatrix} \hat{C} \\ \hat{D} \end{pmatrix}$ für das Modell $b = C + Dt$ zu finden, geht man wie folgt vor:

\begin{enumerate}
	\item \textbf{Datenmatrix $A$ aufstellen:} \\
	Die erste Spalte besteht aus Einsen (für den Achsenabschnitt $C$), die zweite aus den $t$-Werten.
	$$ A = \begin{pmatrix} 1 & t_1 \\ 1 & t_2 \\ \vdots & \vdots \\ 1 & t_m \end{pmatrix} $$
	
	\item \textbf{Zielvektor $b$ aufstellen:} \\
	Enthält die beobachteten Werte.
	$$ b = \begin{pmatrix} b_1 \\ b_2 \\ \vdots \\ b_m \end{pmatrix} $$
	
	\item \textbf{Lösen:} \\
	Berechne $\hat{x}$ mittels der Normalengleichung:
	$$ \hat{x} = (A^T A)^{-1} A^T b $$
	Der resultierende Vektor enthält die gesuchten Regressionskoeffizienten $\hat{C}$ und $\hat{D}$.
\end{enumerate}
\chapter{Multiple lineare Regression}

Kommt bald...

\chapter{Eigenwerte und Eigenvektoren}

Sicher hinzufügen wie man einfach eigenwerte ablesen kann und auch eigenvektore



\chapter{Eigenwerte- und Eigenvektoren}

\section{Positiv definite Matrizen}

\textbf{Definition} \\
Eine symmetrische Matrix $A$ heisst \textbf{positiv definit}, wenn für jeden Vektor $x \neq 0$ gilt:
$$
x^T A x > 0
$$
Gilt nur $x^T A x \geq 0$, nennt man die Matrix \textbf{positiv semi-definit}.
\vspace{1em}
\par 
\noindent
\textbf{Regeln (Symmetrische Matrizen):}
\par % Erzwingt einen Absatz nach "Regeln"
\vspace{1em} % Fügt einen definierten Abstand ein (wie vorher)
\noindent % Einzugsschutz
\begin{tabular}[t]{@{} p{0.45\textwidth} | p{0.45\textwidth} @{}} 
    
    % LINKER BLOCK (Zelle 1)
    \textbf{Standard Regeln}\par
    \begin{itemize}[leftmargin=*, itemsep=2pt, parsep=0pt, topsep=0pt]
        \item Die Matrix ist positiv definit, wenn alle \textbf{Eigenwerte $\lambda$ positiv} sind ($\lambda > 0$).
        \item Für symmetrische $2 \times 2$ Matrix gilt:
        $$ a > 0 \quad \text{und} \quad \det(A) = ad - b^2 > 0 $$
        $A = \begin{pmatrix} a & b \\ b & d \end{pmatrix}$ 
    \end{itemize}
    & % <- Trennzeichen zwischen den Spalten
    
    % RECHTER BLOCK (Zelle 2)
    \textbf{Sylvester-Kriterium (min 1 Krt.)}
    \begin{enumerate}[itemsep=0pt, parsep=0pt, topsep=0pt]
        \item Alle Pivots sind positiv.
        \item Alle führenden Hauptminoren ($\det$ oben links) sind positiv.
        \item Alle Eigenwerte $\lambda_i > 0$.
        \item Die definierende Eigenschaft gilt: $x^T A x > 0$ für alle $x \neq 0$.
        \item Es existiert eine Zerlegung $A = R^T R$ (mit $R$ mit unabhängigen Spalten).
    \end{enumerate}
    
\end{tabular}
\vspace{1em}


\section{Diagonalisierung}

\textbf{Voraussetzung:}
Die $n \times n$-Matrix $A$ muss $n$ linear unabhängige Eigenvektoren besitzen.

\begin{itemize}[itemsep=2pt, parsep=0pt, topsep=2pt]
    \item \textbf{Hauptformeln:}
    $$ A = V \Lambda V^{-1} \quad \text{bzw.} \quad \Lambda = V^{-1} A V $$
    
    \item \textbf{Ableitung:} Die Grundgleichung $A x_i = \lambda_i x_i$ wird zur Matrixgleichung $AV = V\Lambda$.
    
    \item \textbf{Variablen:}
    \begin{itemize}[itemsep=0pt, parsep=0pt, topsep=2pt]
        \item $V$: Die **Eigenvektormatrix** (Spalten sind die Eigenvektoren $x_i$).
        \item $\Lambda$: Die **Eigenwertmatrix** (Diagonalmatrix mit den Eigenwerten $\lambda_i$).
    \end{itemize}
    \[
A \underbrace{[x_1, x_2]}_{\mathbf{V}} = [\lambda_1 \mathbf{x_1}, \lambda_2 \mathbf{x_2}] = \underbrace{[x_1, x_2]}_{\mathbf{V}} \underbrace{\begin{bmatrix} \lambda_1 & 0 \\ 0 & \lambda_2 \end{bmatrix}}_{\mathbf{\Lambda}}
\]
\end{itemize}

\section{Eigenschaften von Determinante (det) und Spur (tr)}

Es gilt für beliebige Matrizen $\mathbf{A}(n \times n)$:
\begin{itemize}[itemsep=0pt, parsep=0pt, topsep=2pt]
    \item Die Spur der Matrix ist gleich der Summe der Eigenwerte.
    \item Die Determinante der Matrix ist gleich dem Produkt der Eigenwerte.
\end{itemize}

Für beliebige Matrizen $\mathbf{A}(n \times n)$ und $\mathbf{B}(n \times n)$ und ein Skalar $c$ gilt:
\begin{align*}
\mathrm{tr}(\mathbf{A} + \mathbf{B}) &= \mathrm{tr}(\mathbf{A}) + \mathrm{tr}(\mathbf{B}) \\
\mathrm{tr}(c\mathbf{A}) &= c \cdot \mathrm{tr}(\mathbf{A}) \\
\det(c\mathbf{A}) &= c^n \det(\mathbf{A}) \\
\mathrm{tr}(\mathbf{A}\mathbf{B}) &= \mathrm{tr}(\mathbf{B}\mathbf{A}) \\
\det(\mathbf{A}\mathbf{B}) &= \det(\mathbf{B}\mathbf{A}) \\
\det(\mathbf{A}\mathbf{B}) &= \det(\mathbf{A}) \det(\mathbf{B}) \\
\det(\mathbf{A}^{-1}) &= (\det(\mathbf{A}))^{-1}
\end{align*}

\section{Ähnliche Matrizen}

\subsection*{Definition & Kern-Eigenschaften}
\begin{itemize}[itemsep=2pt, parsep=0pt, topsep=2pt]
    \item \textbf{Definition:} Zwei Matrizen $\mathbf{A}$ und $\mathbf{B}$ sind ähnlich, wenn eine invertierbare Matrix $\mathbf{M}$ existiert, sodass gilt:
    $$ \mathbf{B} = \mathbf{M}^{-1}\mathbf{A}\mathbf{M} $$
    \item \textbf{Theorem:} Ähnliche Matrizen haben immer die \textbf{selben Eigenwerte} ($\lambda$).
    \item \textbf{Folgerung:} Haben dieselbe \textbf{Spur} ($\text{tr} = \sum \lambda$) und \textbf{Determinante} ($\det = \prod \lambda$).
\end{itemize}

\subsection*{Prüfung und Beispiel}
\begin{itemize}[itemsep=2pt, parsep=0pt, topsep=2pt]
    \item Bei Diagonalmatrizen können die Eigenwerte $\lambda$ direkt von der Diagonale abgelesen werden.
    \item \textbf{Beispiel-Matrizen} (alle ähnlich mit $\lambda_1=2, \lambda_2=4$):
    $$ \begin{bmatrix} 2 & 3 \\ 0 & 4 \end{bmatrix}, \quad \begin{bmatrix} 2 & 0 \\ 0 & 4 \end{bmatrix}, \quad \begin{bmatrix} 0 & 2 \\ -4 & 6 \end{bmatrix} $$
\end{itemize}

\subsection*{Nachweis der Ähnlichkeit durch Matrix M}

\textbf{Ziel:} Finde eine invertierbare Matrix $\mathbf{M}$, die $\mathbf{B} = \mathbf{M}^{-1}\mathbf{A}\mathbf{M}$ erfüllt.

\begin{enumerate}[itemsep=2pt, parsep=0pt, topsep=2pt]
    \item \textbf{Prüfbedingung aufstellen:} Nutze die äquivalente Gleichung:
    $$ \mathbf{M}\mathbf{B} = \mathbf{A}\mathbf{M} $$
    
    \item \textbf{Transformationsmatrix $\mathbf{M}$ bestimmen (Heuristik):}
    Wähle $\mathbf{M}$ basierend auf der visuellen Transformation (Vorzeichen, Vertauschung) zwischen $\mathbf{A}$ und $\mathbf{B}$. Beispielwahl: $\mathbf{M} = \begin{bmatrix} 0 & 1 \\ -1 & 0 \end{bmatrix}$.
    
    \item \textbf{Gleichung verifizieren (Berechnung):} Die Multiplikation beider Seiten muss zum gleichen Ergebnis führen.
    
    \begin{align*}
        \mathbf{MB} &= \begin{bmatrix} 0 & 1 \\ -1 & 0 \end{bmatrix} \begin{bmatrix} 1 & -1 \\ -1 & 1 \end{bmatrix} = \begin{bmatrix} -1 & 1 \\ -1 & 1 \end{bmatrix} \\
        \mathbf{AM} &= \begin{bmatrix} 1 & 1 \\ 1 & 1 \end{bmatrix} \begin{bmatrix} 0 & 1 \\ -1 & 0 \end{bmatrix} = \begin{bmatrix} -1 & 1 \\ -1 & 1 \end{bmatrix}
    \end{align*}
    
    \textbf{Fazit:} Da $\mathbf{MB} = \mathbf{AM}$ gilt, ist die Ähnlichkeit nachgewiesen.
\end{enumerate}

\section{Potenzen von Matrizen}

\subsection*{Eigenschaften der Potenzierung}
\begin{itemize}[itemsep=2pt, parsep=0pt, topsep=2pt]
    \item Falls $\mathbf{A}$ potenziert wird, bleiben die Eigenvektoren $x$ gleich, die Eigenwerte $\lambda$ werden aber potenziert ($\lambda \to \lambda^n$).
    \item Für jeden EV $x$ zum EW $\lambda$ gilt:
    $$ \mathbf{A}^n x = \lambda^n x $$
\end{itemize}

\subsection*{Effiziente Berechnung von $\mathbf{A}^n$}
\begin{itemize}[itemsep=2pt, parsep=0pt, topsep=2pt]
    \item \textbf{Methode:} Berechnung mittels Diagonalisierung $\mathbf{A} = \mathbf{V} \mathbf{\Lambda} \mathbf{V}^{-1}$.
    \item \textbf{Endformel:}
    $$ \mathbf{A}^n = \mathbf{V} \mathbf{\Lambda}^n \mathbf{V}^{-1} $$
    \item \textbf{Vorteil (2x2):} Die Potenzierung von $\mathbf{\Lambda}$ erfolgt trivial auf der Diagonale:
    $$ \mathbf{\Lambda}^n = \begin{pmatrix} \lambda_1^n & 0 \\ 0 & \lambda_2^n \end{pmatrix} $$
\end{itemize}
\chapter{Singulärwertzerlegung-I}

\section{Definition}

\textbf{Satz der Singulärwertzerlegung} \\
Die Singulärwertzerlegung (SVD) zerlegt jede beliebige $m \times n$-Matrix $A$ in ein Produkt von drei speziellen Matrizen[cite: 66]:
$$
A = U \Sigma V^T
$$
Die Eigenschaften der Komponenten sind:
\begin{itemize}
	\item \textbf{$U$:} Eine orthogonale $m \times m$-Matrix ($U^T U = I$, $U^{-1} = U^T$)[cite: 69]. Die Spalten von $U$ heissen \textbf{Links-Singulärvektoren}[cite: 268].
	\item \textbf{$\Sigma$:} Eine $m \times n$-Diagonalmatrix. Die Einträge auf der Diagonalen heissen \textbf{Singulärwerte} $\sigma_i$[cite: 70]. Sie sind positiv und der Grösse nach geordnet ($\sigma_1 \ge \sigma_2 \ge \dots \ge \sigma_n \ge 0$)[cite: 152].
	\item \textbf{$V$:} Eine orthogonale $n \times n$-Matrix ($V^T V = I$, $V^{-1} = V^T$)[cite: 71]. Die Spalten von $V$ heissen \textbf{Rechts-Singulärvektoren}[cite: 268].
\end{itemize}

\textbf{Zusammenhang mit symmetrischen Matrizen} \\
Die Berechnung der SVD basiert auf den Eigenwerten und Eigenvektoren der symmetrischen Matrizen $A^T A$ und $A A^T$[cite: 42]:
\begin{itemize}
	\item $A^T A = V \Sigma^2 V^T$: Die Matrix $A^T A$ liefert die Rechts-Singulärvektoren $V$ und die Quadrate der Singulärwerte[cite: 270, 271].
	\item $A A^T = U \Sigma^2 U^T$: Die Matrix $A A^T$ liefert die Links-Singulärvektoren $U$[cite: 272, 273].
\end{itemize}

\section{Berechnung}

Die Bestimmung der SVD erfolgt typischerweise in folgenden Schritten:

\textbf{1. Berechnung von $V$ und $\Sigma$}
\begin{enumerate}
	\item Berechne die symmetrische Matrix $A^T A$[cite: 269].
	\item Bestimme die Eigenwerte $\lambda_i$ von $A^T A$ durch Lösen der charakteristischen Gleichung $\det(A^T A - \lambda I) = 0$[cite: 287].
	\item Die \textbf{Singulärwerte} sind die Wurzeln dieser Eigenwerte: $\sigma_i = \sqrt{\lambda_i}$[cite: 280, 288]. Damit ist die Matrix $\Sigma$ bestimmt.
	\item Bestimme die zugehörigen normierten Eigenvektoren. Diese bilden die Spalten $v_i$ der Matrix $V$[cite: 281].
\end{enumerate}

\textbf{2. Berechnung von $U$}
Um die Matrix $U$ zu bestimmen, nutzt man die Beziehung $A V = U \Sigma$, was spaltenweise $A v_i = \sigma_i u_i$ entspricht[cite: 321, 332].
\begin{itemize}
	\item Für jeden Singulärwert $\sigma_i > 0$ berechnet sich der Vektor $u_i$ durch:
	$$
	u_i = \frac{1}{\sigma_i} A v_i
	$$
	Dies stellt sicher, dass die Vorzeichen korrekt sind[cite: 318].
	\item Falls $A$ nicht quadratisch ist oder $\sigma_i = 0$, müssen die fehlenden Vektoren $u_i$ so gewählt werden, dass sie orthonormal zu den bestehenden Spalten sind (Basisergänzung)[cite: 383].
\end{itemize}

\textbf{Alternativer Weg für $U$:} \\
Man kann $U$ auch als Eigenvektoren von $A A^T$ berechnen[cite: 307]. Dabei muss jedoch darauf geachtet werden, dass die Vorzeichen mit $V$ kompatibel sind.
\chapter{Singulärwertzerlegung II}

To do...



% ====================================================================
\end{document}
% ====================================================================