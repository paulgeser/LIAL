\chapter{Eigenwerte und Eigenvektoren}

\section{Definition und Berechnung}

\textbf{Eigenwertproblem} \\
Gesucht sind Skalare $\lambda$ (Eigenwerte) und Vektoren $x \neq 0$ (Eigenvektoren), die für eine quadratische Matrix $A$ die Gleichung erfüllen:
$$
Ax = \lambda x \quad \Longleftrightarrow \quad (A - \lambda I)x = 0
$$

\textbf{1. Eigenwerte bestimmen} \\
Die Eigenwerte $\lambda$ sind die Lösungen der \textbf{charakteristischen Gleichung}:
$$
\det(A - \lambda I) = 0
$$

\textbf{2. Eigenvektoren bestimmen} \\
Für jeden Eigenwert $\lambda_i$ wird der zugehörige Eigenvektor $x_i$ durch Lösen des homogenen linearen Gleichungssystems ermittelt:
$$
(A - \lambda_i I)x_i = 0
$$

\section{Vollständiges Beispiel}
\textbf{Matrix:} $ A = \begin{pmatrix} 2 & 1 \\ 1 & 2 \end{pmatrix} $

\textbf{1. Eigenwerte:} \\
Die charakteristische Gleichung lautet:
$$ \det(A - \lambda I) = \det \begin{pmatrix} 2-\lambda & 1 \\ 1 & 2-\lambda \end{pmatrix} = (2-\lambda)^2 - 1 = \lambda^2 - 4\lambda + 3 = 0 $$
Lösungen (Faktorisierung $(\lambda - 3)(\lambda - 1) = 0$):
$$ \lambda_1 = 3, \quad \lambda_2 = 1 $$

\textbf{2. Eigenvektoren:}
\begin{itemize}
    \item \textbf{Zu $\lambda_1 = 3$:}
    $$ (A - 3I)x = \begin{pmatrix} -1 & 1 \\ 1 & -1 \end{pmatrix} \begin{pmatrix} x_1 \\ x_2 \end{pmatrix} = \begin{pmatrix} 0 \\ 0 \end{pmatrix} \implies -x_1 + x_2 = 0 \implies x_1 = \begin{pmatrix} 1 \\ 1 \end{pmatrix} $$
    
    \item \textbf{Zu $\lambda_2 = 1$:}
    $$ (A - 1I)x = \begin{pmatrix} 1 & 1 \\ 1 & 1 \end{pmatrix} \begin{pmatrix} x_1 \\ x_2 \end{pmatrix} = \begin{pmatrix} 0 \\ 0 \end{pmatrix} \implies x_1 + x_2 = 0 \implies x_2 = \begin{pmatrix} 1 \\ -1 \end{pmatrix} $$
\end{itemize}

\section{Eigenschaften und Spektralzerlegung}

\textbf{Spur und Determinante:}
\begin{itemize}
    \item $\text{spur}(A) = \sum_{i=1}^{n} a_{i,i} = \sum_{i=1}^{n} \lambda_i$
    \item $\det(A) = \prod_{i=1}^{n} \lambda_i$
\end{itemize}

\textbf{Symmetrische Matrizen ($A = A^T$):}
\begin{itemize}
    \item Alle Eigenwerte $\lambda$ sind reell.
    \item Eigenvektoren zu verschiedenen $\lambda$ sind orthogonal.
\end{itemize}

\textbf{Spektralzerlegung (für symmetrische Matrizen):} \\
Sei $V$ die orthogonale Matrix der (normierten) Eigenvektoren $x_i$ und $\Lambda$ die Diagonalmatrix der Eigenwerte $\lambda_i$.
\begin{itemize}
    \item \textbf{Matrixform:} $ A = V\Lambda V^T $
    \item \textbf{Summenform:} $ A = \sum_{i=1}^{n} \lambda_i x_i x_i^T $
\end{itemize}