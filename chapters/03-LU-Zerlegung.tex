\chapter{LU-Zerlegung}

\section{Invertierbare Matrizen}

\textbf{Definition} \\
Eine quadratische $n \times n$-Matrix $A$ heisst \textbf{invertierbar}, \textbf{regulär} oder \textbf{nichtsingulär}, wenn eine $n \times n$-Matrix $B$ existiert, sodass gilt:
$$
A \cdot B = B \cdot A = E
$$
Man nennt $B$ die zu $A$ inverse Matrix und bezeichnet sie mit $B = A^{-1}$.

\textbf{Eigenschaften}
\begin{itemize}
	\item Ist $A$ invertierbar, hat das Gleichungssystem $Ax = b$ für jedes $b$ eine eindeutige Lösung ($x = A^{-1}b$).
	\item Das Inverse eines Produkts kehrt die Reihenfolge um: $(AB)^{-1} = B^{-1} A^{-1}$.
\end{itemize}

\section{Permutationsmatrix}

\textbf{Definition} \\
Eine Permutationsmatrix $P$ ist eine quadratische Matrix, bei der jede Zeile und jede Spalte genau eine Eins enthält, wobei alle anderen Elemente Null sind. Sie entsteht aus der Einheitsmatrix durch Vertauschen von Zeilen.

\textbf{Wirkung}
\begin{itemize}
	\item \textbf{Multiplikation von links ($PA$):} Vertauscht die Zeilen von $A$.
	\item \textbf{Multiplikation von rechts ($AP$):} Vertauscht die Spalten von $A$.
\end{itemize}

\textbf{Beispiel (Zeilentausch)} \\
Um bei einer $3 \times 3$-Matrix die 1. und 2. Zeile zu vertauschen, wird die Einheitsmatrix entsprechend umgeformt (Tausch der 1. und 2. Zeile von $E$):
$$
P = \begin{pmatrix} 0 & 1 & 0 \\ 1 & 0 & 0 \\ 0 & 0 & 1 \end{pmatrix}
$$
Angewandt auf eine Matrix $A$:
$$
P \cdot A = \begin{pmatrix} 0 & 1 & 0 \\ 1 & 0 & 0 \\ 0 & 0 & 1 \end{pmatrix} \cdot \begin{pmatrix} 0 & 1 & 1 \\ 1 & 2 & 1 \\ 2 & 7 & 9 \end{pmatrix} = \begin{pmatrix} 1 & 2 & 1 \\ 0 & 1 & 1 \\ 2 & 7 & 9 \end{pmatrix}
$$
Hier wurde die erste Zeile $(0, 1, 1)$ mit der zweiten Zeile $(1, 2, 1)$ vertauscht.

\section{LR-Zerlegung (LU Decomposition)}

\textbf{Konzept} \\
Die LR-Zerlegung faktorisiert eine Matrix $A$ in das Produkt einer unteren Dreiecksmatrix $L$ (Lower) und einer oberen Dreiecksmatrix $U$ (Upper).
$$
A = L \cdot U
$$
\begin{itemize}
	\item \textbf{$U$ (Upper):} Ist die Zeilenstufenform von $A$, die durch Gauss-Elimination entsteht.
	\item \textbf{$L$ (Lower):} Ist eine untere Dreiecksmatrix mit Einsen auf der Diagonalen. Sie enthält die Multiplikatoren der Eliminationsschritte ($l_{ij}$ an der Position $i$-te Zeile, $j$-te Spalte).
\end{itemize}

\textbf{Zerlegung mit Zeilentausch} \\
Müssen während der Elimination Zeilen vertauscht werden (Pivotisierung), wird die Matrix $PA$ zerlegt:
$$
P \cdot A = L \cdot U
$$

\section{Lösen eines LGS mittels LR-Zerlegung}

Das Verfahren nutzt die Zerlegung $A = LU$, um das System $Ax = b$ durch zwei einfachere Dreieckssysteme zu lösen. Dies ist besonders effizient, wenn mehrere Gleichungssysteme mit derselben Matrix $A$, aber unterschiedlichen Vektoren $b$ gelöst werden müssen.

\textbf{Vorgehen}
Das System $Ax = b$ wird umgeformt zu $L(Ux) = b$. Man substituiert $Ux = y$ und löst in zwei Schritten:
\begin{enumerate}
	\item \textbf{Vorwärtseinsetzen:} Löse $L y = b$ nach $y$ auf.
	\item \textbf{Rückwärtseinsetzen:} Löse $U x = y$ nach $x$ auf.
\end{enumerate}

\textbf{1. Vorwärtseinsetzen ($Ly=b$)} \\
Da $L$ eine untere Dreiecksmatrix ist, lassen sich die Unbekannten $y$ von oben nach unten berechnen.
Beispiel für $3 \times 3$:
$$
\begin{pmatrix} 1 & 0 & 0 \\ l_{21} & 1 & 0 \\ l_{31} & l_{32} & 1 \end{pmatrix} \begin{pmatrix} y_1 \\ y_2 \\ y_3 \end{pmatrix} = \begin{pmatrix} b_1 \\ b_2 \\ b_3 \end{pmatrix}
$$
Daraus ergeben sich direkt die Gleichungen:
\begin{align*}
	y_1 &= b_1 \\
	y_2 &= b_2 - l_{21}y_1 \\
	y_3 &= b_3 - l_{31}y_1 - l_{32}y_2
\end{align*}

\textbf{2. Rückwärtseinsetzen ($Ux=y$)} \\
Da $U$ eine obere Dreiecksmatrix ist, lassen sich die Unbekannten $x$ von unten nach oben berechnen.
Beispiel für $3 \times 3$:
$$
\begin{pmatrix} u_{11} & u_{12} & u_{13} \\ 0 & u_{22} & u_{23} \\ 0 & 0 & u_{33} \end{pmatrix} \begin{pmatrix} x_1 \\ x_2 \\ x_3 \end{pmatrix} = \begin{pmatrix} y_1 \\ y_2 \\ y_3 \end{pmatrix}
$$
Daraus ergeben sich die Gleichungen (Auflösen nach $x_3$, dann $x_2$, dann $x_1$):
\begin{align*}
	x_3 &= \frac{y_3}{u_{33}} \\
	x_2 &= \frac{y_2 - u_{23}x_3}{u_{22}} \\
	x_1 &= \frac{y_1 - u_{12}x_2 - u_{13}x_3}{u_{11}}
\end{align*}

\textbf{Konkretes Beispiel ($2 \times 2$)} \\
Gegeben sei das System $Ax = b$ mit $A = \begin{pmatrix} 1 & 2 \\ 4 & 9 \end{pmatrix}$ und $b = \begin{pmatrix} 5 \\ 21 \end{pmatrix}$. \\
Die LR-Zerlegung von $A$ ergibt:
$L = \begin{pmatrix} 1 & 0 \\ 4 & 1 \end{pmatrix}, \quad U = \begin{pmatrix} 1 & 2 \\ 0 & 1 \end{pmatrix}$.

\begin{itemize}
	\item \textbf{Schritt 1 ($Ly = b$):}
	$$ \begin{pmatrix} 1 & 0 \\ 4 & 1 \end{pmatrix} \begin{pmatrix} y_1 \\ y_2 \end{pmatrix} = \begin{pmatrix} 5 \\ 21 \end{pmatrix} $$
	$$ y_1 = 5 $$
	$$ 4 \cdot 5 + y_2 = 21 \implies y_2 = 1 $$
	$\implies y = \begin{pmatrix} 5 \\ 1 \end{pmatrix}$
	
	\item \textbf{Schritt 2 ($Ux = y$):}
	$$ \begin{pmatrix} 1 & 2 \\ 0 & 1 \end{pmatrix} \begin{pmatrix} x_1 \\ x_2 \end{pmatrix} = \begin{pmatrix} 5 \\ 1 \end{pmatrix} $$
	$$ x_2 = 1 $$
	$$ x_1 + 2 \cdot 1 = 5 \implies x_1 = 3 $$
	$\implies \textbf{Lösung: } x = \begin{pmatrix} 3 \\ 1 \end{pmatrix}$
\end{itemize}