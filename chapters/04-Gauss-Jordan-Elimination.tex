\chapter{Gauss-Jordan-Elimination}

\section{Gauss-Jordan-Algorithmus}

\textbf{Konzept} \\
Das Gauss-Jordan-Verfahren ist eine Erweiterung des Gauss-Eliminationsverfahrens. Ziel ist es, die Matrix nicht nur in die Zeilenstufenform, sondern in die \textbf{reduzierte Zeilenstufenform} (Diagonalmatrix, idealerweise Einheitsmatrix) zu bringen.

\textbf{Vorgehen}
\begin{enumerate}
	\item Matrix in Zeilenstufenform bringen (Gauss-Elimination).
	\item Von \textbf{rechts nach links} und von \textbf{unten nach oben} Nullen oberhalb der Pivotelemente erzeugen.
	\item Pivotelemente durch Division auf $1$ normieren.
\end{enumerate}

\textbf{Anwendung}
\begin{itemize}
	\item Direktes Ablesen der Lösungen ohne Rückwärtseinsetzen.
	\item Simultanes Lösen mehrerer Gleichungssysteme ($Ax = a, Ax = b, \dots$) durch Erweiterung der Matrix.
	\item Berechnung der inversen Matrix.
\end{itemize}

\section{Inverse Matrix}

\textbf{Definition} \\
Eine quadratische Matrix $A$ ist invertierbar, wenn eine Matrix $A^{-1}$ existiert, für die gilt:
$$ A \cdot A^{-1} = I \quad \text{und} \quad A^{-1} \cdot A = I $$
wobei $I$ die Einheitsmatrix ist.

\textbf{Berechnung mittels Gauss-Jordan} \\
Man schreibt die Matrix $A$ und die Einheitsmatrix $I$ nebeneinander: $(A \mid I)$. Durch Anwendung des Gauss-Jordan-Algorithmus wird die linke Seite zur Einheitsmatrix umgeformt. Die rechte Seite wird dabei automatisch zur Inversen.
$$
(A \mid I) \xrightarrow{\text{Gauss-Jordan}} (I \mid A^{-1})
$$

\section{Determinante}

\textbf{Definition und Bedeutung} \\
Die Determinante ist eine Kennzahl (Skalar) für quadratische Matrizen.
\begin{itemize}
	\item $\det(A) \neq 0 \iff A$ ist regulär (invertierbar).
	\item $\det(A) = 0 \iff A$ ist singulär (nicht invertierbar).
	\item Geometrisch entspricht sie im $\mathbb{R}^2$ der Fläche und im $\mathbb{R}^3$ dem Volumen (Spatprodukt).
\end{itemize}

\textbf{Berechnung}
\begin{itemize}
	\item \textbf{$2 \times 2$ Matrix:}
	$$ A = \begin{pmatrix} a & b \\ c & d \end{pmatrix} \implies \det(A) = ad - bc $$
	
	\item \textbf{$3 \times 3$ Matrix (Regel von Sarrus):} \\
	Für $A = \begin{pmatrix} u_1 & v_1 & w_1 \\ u_2 & v_2 & w_2 \\ u_3 & v_3 & w_3 \end{pmatrix}$ gilt:
	$$ \det(A) = (u_1 v_2 w_3 + v_1 w_2 u_3 + w_1 u_2 v_3) - (u_3 v_2 w_1 + v_3 w_2 u_1 + w_3 u_2 v_1) $$
	
	\item \textbf{Dreiecksmatrizen:} \\
	Die Determinante ist das Produkt der Diagonalelemente.
\end{itemize}

\textbf{Rechenregeln für Determinanten}
\begin{itemize}
	\item \textbf{Zeilenvertauschung:} Das Vorzeichen ändert sich ($\det_{neu} = - \det_{alt}$).
	\item \textbf{Zeilenmultiplikation:} Wird eine Zeile mit Faktor $k$ multipliziert, ändert sich die Determinante um Faktor $k$.
	\item \textbf{Zeilenaddition:} Die Addition des Vielfachen einer Zeile zu einer anderen ändert die Determinante \textbf{nicht}.
	\item \textbf{Transposition:} $\det(A) = \det(A^T)$.
\end{itemize}

\section{Cramersche Regel}

Ein Verfahren zum Lösen linearer Gleichungssysteme $Ax = b$, wenn $\det(A) \neq 0$.

\textbf{Formel} \\
Die Lösung für die Unbekannte $x_i$ ist:
$$
x_i = \frac{\det(A_i)}{\det(A)}
$$
Dabei ist $A_i$ die Matrix, die entsteht, wenn man die $i$-te Spalte von $A$ durch den Vektor $b$ ersetzt.

\textbf{Beispiel ($2 \times 2$)}
$$ x = \frac{D_x}{D}, \quad y = \frac{D_y}{D} $$
mit $D = \det(A)$, $D_x = \det(b, \text{Spalte}_2)$, $D_y = \det(\text{Spalte}_1, b)$.
